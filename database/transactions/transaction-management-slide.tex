\documentclass[xcolor={table, dvipsnames}]{beamer}
\usepackage[utf8]{inputenc,vietnam}

\usepackage{utopia} %font utopia imported
\usetheme{Madrid}
\usecolortheme{default}

% Code syntax highlighter
% https://latex-beamer.com/tutorials/beamer-code/
% Install required tool https://pygments.org/
% Add to PATH: texmaker > Preference
% e.g. /Users/minhnt/Library/Python/3.8/bin
\usepackage{minted}

\usepackage{hyperref}
\hypersetup{
    colorlinks=true,
    linkcolor=blue,
    filecolor=magenta,      
    urlcolor=cyan,
}

\urlstyle{same}

\newcommand\boldblue[1]{\textcolor{blue}{\textbf{#1}}}

%------------------------------------------------
%Information to be included in the title page:
\title{Transaction Management}

\subtitle{Hiểu và sử dụng transaction trong bài toán Concurrency}

%
%\author{Nguyen The Minh}
%\institute{Electronics and Computer Science Home}
%\date{Computing Conference, Oct 2050}
%

%\logo{\includegraphics[height=1.5cm]{atom.png}}

%End of title page
%-------------------------------------------------

%------------------------------------------------------------
%The next block of commands puts the table of contents at the 
%beginning of each section and highlights the current section:

\AtBeginSection[]
{
  \begin{frame}
    \frametitle{Nội dung}
    \tableofcontents[currentsection]
  \end{frame}
}
%------------------------------------------------------------

\begin{document}

\frame{\titlepage}

%---------------------------------------------------------
%This block of code is for the table of contents after
%the title page
\begin{frame}
\frametitle{Nội dung}
\tableofcontents
\end{frame}
%---------------------------------------------------------

\section{Giới thiệu Transactions}

\begin{frame}
\frametitle{Giới thiệu về Transactions}

Nội dụng bao gồm:
\begin{itemize}
\item Khái niệm Transaction
\item Tính ACID
\item Xử lý đồng thời Transactions
\end{itemize}

\end{frame}

\begin{frame}
\frametitle{Khái niệm Transaction}

Bài toán \textbf{Concurrency Control}:
\begin{itemize}
\item Nhiều người cùng truy cập database tại một thời điểm.
\item Một người cập nhật trong khi có người khác đọc cùng một dữ liệu?
\item Cả hai người cùng cập nhật dữ liệu cùng một lúc?
\end{itemize}
Vậy làm thế nào để truy cập dữ liệu \textbf{hiệu quả} và \textbf{an toàn}?

\end{frame}

\begin{frame}
\frametitle{Khái niệm Transaction}

Transaction:
\begin{itemize}
\item Chuỗi các operations được executed như một \textit{single, logical, atomic} unit
\item Kết quả trung gian không ảnh hưởng đến transactions khác.
\item Hoàn thành tất cả hoặc không gì cả - all-or-nothing.
\end{itemize}
Các thao tác với database sẽ được thực hiện thông quan transactions.
\end{frame}

\begin{frame}
\frametitle{Tính ACID}
4 tính chất của transaction:
\begin{itemize}
\item \textbf{Atomic}: hoặc là tất cả hoặc không.
\item \textbf{Consistency}: đảm bảo tính toàn vẹn, trách nhiệm của người dùng.
\item \textbf{Isolation}: độc lập, không ảnh bởi transaction khác.
\item \textbf{Durability}: khi hoàn thành, đảm bảo thay đổi được lưu lại.
\end{itemize}

\end{frame}

\begin{frame}
\frametitle{Consistency và Isolation}
Chuyển tiền từ tài khoản A sang cho B:
\begin{itemize}
\item Trừ tiền A
\item Cộng tiền B
\item Tiêu chí toàn vẹn: A + B
\end{itemize}
Chương trình lỗi: cộng tiền B thiếu 1 dollar so với trừ A
\end{frame}

\begin{frame}
\frametitle{Consistency và Isolation}
Tính cô lập thể hiện:
\begin{itemize}
\item Đồng thời xử lý = Tuần tự
\item Lập lịch xen kẽ các transactions = Tuần tự thực hiện
\end{itemize}
\end{frame}

\begin{frame}
\frametitle{Atomicity và Durability}
Transaction có thể bị dừng giữa chừng:
\begin{itemize}
\item Bị hủy bởi người dùng
\item Hệ thống bị crash (mất điện)
\item Dữ liệu bất thường hay không truy cập được ổ đĩa
\end{itemize}
Các thao tác giữa chừng cần phải undo?\\
DBMS sử dụng logs để lưu các thay đổi
\begin{itemize}
\item Có thể undo
\item Hay khôi phục hệ thống nếu chưa kịp lưu vào ổ đĩa
\end{itemize}
\end{frame}

\begin{frame}
\frametitle{Lập lịch xử lý transactions}
Xử lý tuần tự vs interleaved (xen kẽ):
\begin{itemize}
\item Tận dụng CPU
\item Tránh chờ I/O
\item Đảm bảo thời gian response
\end{itemize}
\end{frame}

\begin{frame}
\frametitle{Lập lịch xử lý transactions}
Một transaction - một series các actions:
\begin{itemize}
\item Transaction T đọc object O: $R_{T}(O)$
\item T ghi object O: $W_{T}(O)$
\item T complete: $Commit_{T}$
\item T hủy (rollback): $Abort_{T}$
\end{itemize}
Bỏ qua T nếu ngữ cảnh rõ ràng
\end{frame}

\begin{frame}
\frametitle{Ví dụ lịch xử lý transactions}
\begin{tabular}{ll}
T1 & T2 \\ 
\hline 
R(A) &  \\ 
W(A) &  \\ 
 & R(A) \\ 
 & W(A) \\ 
R(B) &  \\ 
W(B) &  \\ 
 & R(B) \\ 
 & W(B) \\ 
 & Commit \\ 
Commit &  \\ 
\end{tabular} 
\end{frame}

\begin{frame}
\frametitle{Xung đột dữ liệu}
Khi có nhiều transactions đồng thời, tồn tại các xung đột:
\begin{itemize}
\item Dirty read: đọc dữ liệu đã thay đổi bởi một transaction khác chưa commit
\item Nonrepeatable read: đọc (sử dụng) cùng một dữ liệu lần 2 nhưng thấy đã bị thay đổi bởi transaction khác
\item Phantom read: chạy lại cùng câu truy vấn với search condition và thấy tập dữ liệu trả về đã bị thay đổi bởi transaction khác
\end{itemize}
\end{frame}

\begin{frame}
\frametitle{Transaction Isolation Level}
\begin{table}
\begin{tabular}{|l|l|l|l|}
\hline 
\textbf{Isolation Level} & \textbf{Dirty} & \textbf{Nonrepeatable} & \textbf{Phantom} \\ 
 & \textbf{Read} & \textbf{Read} & \textbf{Read} \\
\hline 
Read uncommited & X (PG) & X & X \\ 
\hline 
Read commited & -- & X & X \\ 
\hline 
Repeatable read & -- & -- & X (PG) \\ 
\hline 
Serializable & -- & -- & -- \\ 
\hline 
\end{tabular} 
\caption{Bốn mức isolation của transactions}
\end{table}
PG: not in Postgresql
\end{frame}

\begin{frame}
\frametitle{Serializable Level}
Isolation level kiểm soát mức độ isolation của một transaction với các transaction đồng thời khác. Cao nhất là \textit{Serializable}:
\begin{itemize}
\item T chỉ đọc dữ liệu commited (no dirty read)
\item Ko có giá trị đang đọc hay ghi bởi T được thay đổi bởi transaction khác
\item T đang đọc một tập search condition thì tập này không được thay đổi bởi transaction khác
\end{itemize}
\end{frame}

\begin{frame}
\frametitle{REPEATABLE READ Level}
T chỉ đọc dữ liệu commited và ko có giá trị đang đọc hay ghi bởi T được thay đổi bởi transaction khác. Nhưng T có thể gặp phantom read. Ví dụ T truy vấn tất cả Books còn bản copies, transaction khác lại thêm một Book mới. Vậy T bị miss Book mới.
\end{frame}

\begin{frame}
\frametitle{READ COMMITED Level}
T chỉ đọc dữ liệu commited và ko có giá trị đang ghi bởi T được thay đổi bởi transaction khác. Tuy nhiên, một giá trị T đọc có thể bị thay đổi bởi transaction khác, dẫn đến "nonrepeatable read".
\end{frame}

\begin{frame}
\frametitle{READ UNCOMMITED Level}
Ở level này transaction có thể đọc giá trị đang được update bởi transaction (uncommited). Có thể coi là transaction không cần isolation.
\end{frame}

\begin{frame}
\frametitle{Strict Two-phase Locking - Strict 2PL}
Một cách để thực hiện các mức isolation là dùng cơ chế locking - Strict 2PL:
\begin{itemize}
\item T muốn đọc (ghi) O thì phải lấy được \textbf{shared lock} (\textbf{exclusive lock}) của O. Khi có một exclusive lock trên O, các transactions không thể lấy lock shared hay exclusive lock khác trên O mà phải chờ.
\item Tất cả locks được release khi transaction kết thúc.
\end{itemize}
Strict 2PL sẽ đảm bảo mức isolation "serializable", kết quả thực hiện sẽ giống khi làm tuần tự các transactions. Tuy vậy vấn đề deadlock có thể xảy ra?
\end{frame}

% Transaction support in SQL
\section{Transactions trong SQL}
\begin{frame}[fragile]
\frametitle{Transactions trong SQL}
Isolation mặc định: READ COMMITED Level\\
Tạo transaction bằng command \verb|BEGIN| và \verb|COMMIT|:
\rule{\textwidth}{1pt}
\scriptsize
\begin{minted}{sql}
    BEGIN;
    UPDATE accounts SET balance = balance - 100.00
       WHERE name = 'Alice';
    -- etc etc
    COMMIT;
\end{minted}
\rule{\textwidth}{1pt}

\begin{block}{Transaction characters}
\begin{minted}{sql}
START TRANSACTION [ transaction_mode [, ...] ]

where transaction_mode is one of:

    ISOLATION LEVEL { SERIALIZABLE | REPEATABLE READ | READ COMMITTED 
    | READ UNCOMMITTED }
    READ WRITE | READ ONLY
\end{minted}
    
https://www.postgresql.org/docs/current/sql-start-transaction.html\\
https://dev.mysql.com/doc/refman/8.0/en/set-transaction.html
\end{block}
\end{frame}

\begin{frame}
\frametitle{Transaction trong SQL}
DBMS mặc định xử lý mỗi câu SQL (statement) trong một transaction.\\
Bắt đầu bằng $BEGIN$ và kết thúc $COMMIT$ (nếu thành công).\\
% example/implicit-explicit-transaction.sql
\end{frame}

\begin{frame}
\frametitle{Rollback transaction}
Sử dụng command $ROLLBACK$ để hủy transaction.\\
% example/rollback-command.sql
\pause
Nếu transaction bị lỗi thì không thể $COMMIT$\\
% example/error-commit-transaction.sql
\pause
\begin{block}{Exercise 1: make an error transaction}
Add a constraint that doesn't allow any tags with length shorter than 2 characters.\\
Create a transaction that violates the above constraint.
% example/error-transaction-exercise.sql
\end{block}
\end{frame}

\begin{frame}
\frametitle{Ví dụ sử dụng transaction}
Khi nào sử dụng transaction (explicit)?\\
\pause
Một luồng xử lý hoặc là thành công hoặc là thất bại.\\
\pause
\begin{block}{Exercise 2: lấy ví dụ sử dụng transaction}
Chuyển tiền ngân hàng, trừ tài khoản A và cộng vào B.\\
Các thao tác cần được làm trong cùng một transaction.\\
\end{block} 
\end{frame}

\begin{frame}[fragile]
\frametitle{Rollback with SAVEPOINT}
Bỏ qua một phần và commit phần còn lại.\\
\pause
\begin{minted}{sql}
BEGIN;
UPDATE accounts SET balance = balance - 100.00
	WHERE name = 'Alice';
SAVEPOINT my_savepoint;
UPDATE account SET balance = balance + 100.00
	WHERE name = 'Bob';
-- oops ... forget that and use Wally's account
ROLLBACK TO my_savepoint;
UPDATE account SET balance = balance + 100.00
	WHERE name = 'Wally';
COMMIT;
\end{minted}
\end{frame}
\begin{frame}
\frametitle{Ví dụ về SAVEPOINT}
\begin{block}{Exercise 3: savepoint khi thêm tags}
Tạo một transaction thêm nhiều tags.\\
Định nghĩa một savepoint.\\
Thực hiện rollback về savepoint.\\
\end{block}
\pause
\begin{block}{More about savepoint}
Tạo một transaction có nhiều savepoints.\\
Thực hiện rollback về các savepoints.\\
Kết quả sẽ ntn?.
\end{block}
\end{frame}

\begin{frame}
\frametitle{Thời gian trong transaction}
Transaction được coi là Atomicity.\\
Vậy thời gian trong transaction sẽ ra sao?\\
\end{frame}

%\begin{frame}
%\frametitle{Deadlock trong SQL}
%Ví dụ tình huống xảy ra deadlock.\\
%Khi đó DBMS sẽ xử lý ra sao?\\
%\end{frame}

% Locking section
\section{Locking}

\begin{frame}
\frametitle{Cơ chế Locking}

Xin chào

\end{frame}

% Crash recovery section
\section{Crash Recovery}

\begin{frame}
\frametitle{Giới thiệu về Crash Recovery}

Xin chào

\end{frame}

\end{document}